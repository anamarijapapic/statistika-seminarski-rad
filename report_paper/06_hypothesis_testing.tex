\section{Testiranje hipoteza}

Postavljena je hipoteza o očekivanju populacije, te je proveden t-test za testiranje hipoteze.

K\^od je prikazan u ispisu~\ref{testing}.

\begin{lstlisting}[caption={Testiranje hipoteza}, label=testing]
# 5. Testiranje hipoteze
# Npr. testiranje hipoteze o prosjecnom vremenu provedenom na Zavodu za zaposljavanje cekajuci posao
# Koristi se t-test
alpha = 0.05
t_statistic, p_value = stats.ttest_1samp(data['Vrijeme'], popmean=15)

print("\nTestiranje hipoteza:")
print(f"T-statistika: {t_statistic}")
print(f"P-vrijednost: {p_value}")

if p_value < alpha:
    print("Odbacujemo nultu hipotezu.")
else:
    print("Ne mozemo odbaciti nultu hipotezu.")
\end{lstlisting}

Dakle, proveli smo testiranje hipoteza vezano uz srednju vrijednost populacije (\( \mu \)). Hipoteze su sljedeće:

\[
\begin{aligned}
& H_0: \mu = 15 \\
& H_1: \mu \neq 15 \\
\end{aligned}
\]

Rezultati testa su sljedeći:

\[
\begin{aligned}
& \text{T-statistika}:-1.3363623743324509 \\
& \text{P-vrijednost}:0.18244692949663507 \\
\end{aligned}
\]

\newpage

\begin{enumerate}
  \item \textbf{T-statistika:}\\
T-statistika predstavlja mjeru koliko se srednja vrijednost uzorka razlikuje od srednje vrijednosti populacije iz nulte hipoteze, izraženo u standardnim devijacijama. Negativna vrijednost ukazuje na to da je srednja vrijednost uzorka manja od pretpostavljene srednje vrijednosti populacije.
  \item \textbf{P-vrijednost:}\\
P-vrijednost je vjerojatnost da bismo dobili rezultate testa (ili ekstremnije) ako je nulta hipoteza istinita. Ako je p-vrijednost mala (obično manja od odabrane razine značajnosti, npr. 0.05), obično odbacujemo nultu hipotezu. Ako je p-vrijednost visoka, zadržavamo nultu hipotezu.
\end{enumerate}

Budući da je p-vrijednost (0.182) veća od odabrane razine značajnosti $\alpha = 0.05$, ne možemo odbaciti nultu hipotezu. To sugerira da nema dovoljno statističkih dokaza da se tvrdi da je očekivanje populacije različito od 15 dana.